%%
%  This is all that frontmatter stuff
%
%  This way I can 'not' include it easily

\begin{thesistitlepage}               %% Generate the title page.
\end{thesistitlepage}

\begin{thesisabstract}
Webcams are pretty fucking awesome.  Here's how to visualize them...
\end{thesisabstract}

\begin{thesisacknowledgments}

A special thanks goes to the many graduate students and distinguished faculty
within my department who have reviewed this thesis and helped support the
related research.
\end{thesisacknowledgments}

\begin{thesisdedicationpage}                %% Generate the dedication page.
\end{thesisdedicationpage}

\begin{singlespace}
\tableofcontents


\listoftables

\listoffigures
\end{singlespace}

\chapter{Preface}

This guide contains the School of Engineering's rules for formatting theses and
dissertations.\footnote{Throughout this guide, the word thesis refers to both
theses and dissertations.}   Departments, advisors, and committees may impose
additional rules.  In the past, students were required to study a similar (but
much longer) set of rules and apply them to their theses.  The Association of
Graduate Engineering Students (i.e., AGES) has helped to prepare templates and
style files that simplify thesis preparation.  These files have been set up to
produce acceptably formatted theses and dissertations using several popular
word processing and text formatting programs.  There should be one available in
Microsoft WORD and another in \LaTeX{}.  Students can retrieve these files and
their accompanying instructions from the Engineering Student Services' main web
page.  Check with Engineering Student Servcies (Lopata Hall, Room 303) if you
have any questions.  Students who create their own templates or style files are
invited to submit these files for future use by others.  This guide you are
now reading can be downloaded (in either MS WORD formatted version or a \LaTeX{}
version) and can be utilized as a template for formatting your own theses.  In
short, the margin settings, pagination, table of contents logic, etc. are
already established in the downloadable versions.  You can simply replace the
text within the template with your own text, thereby saving you much setup
time.  \textbf{NOTE:} \uline{This preface page is optional.  A preface page
is usually used to explain further details surrounding the background and
motivation for the work.  You can remove it completely}, but then be sure the
reference to this page is also removed from the Table of Contents.  The
majority of students do not include a preface page.

%%
%% For List of Abbreviations, Glossary or Nomenclature also
%% use \chapter, but put some kind of list environment inside.



%%% Local Variables: 
%%% mode: latex
%%% TeX-master: "thesis-main"
%%% End: 

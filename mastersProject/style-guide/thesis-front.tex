%%
%  This is all that frontmatter stuff
%
%  This way I can 'not' include it easily

% NOTE: do not put any text in the thesistitlepage, thesiscopyrightpage,
% or thesisdedicationpage sections.  If you want to use these pages, then you
% should remove the notes below (e.g., by uncommenting the \iffalse
% and \fi lines) and change the appropriate fields in thesis-main.tex.
% This will ensure that the copyright and dedication lines are positioned
% and formatted correctly.  Additionally, remove the
% thesisacknowledgmentpostscript and listoftablespostscript sections, since
% these are used to add explanatory notes which shouldn't be there in normal
% theses.

\begin{thesistitlepage}               %% Generate the title page.
%\iffalse
\begin{singlespace}
\tiny
{\small \textbf{How to use this document:}}
This sample document outlines guidelines for the proper formatting of theses
and dissertations for Master's and D.Sc.\ degree seeking students within the
School of Engineering at Washington University.  (Ph.D.\ students can also make
use of this document; see special note below.)  This document is formatted
using the same guidelines which it describes.  Consequently, by making an extra
copy of this document you can use it as a template into which you can insert
your own thesis or dissertation textual matter, replacing the original text
with your own while still retaining the general formatting contained within.
This document/template can be downloaded (as either a Microsoft WORD document
OR as a set of \LaTeX{} files) from the Engineering Student Services' web site
and is located with other engineering graduate forms and guides.

{\small \textbf{Important reminders of what needs to be updated:}}
Be certain to use your own full name wherever appropriate.  After removing
these comments, be sure to vertically center the information on this title page
to assure an equal amount of ``white space'' exists both above and below your
title page information.  Your examination committee will likely contain only
three members if you are a Master�s student, as shown in the sample above.
However, D.Sc.\ committees will typically have five members, and Ph.D.\
committees will have six.  Make sure you use the month and year your degree is
officially to be \uline{earned} on the title page, abstract page, and on any
vita page included.  \uline{If this is for your doctoral degree (i.e., either
D.Sc.\ or Ph.D.),} be sure to change all occurrences of the word ``thesis'' to
display as ``dissertation'', and change ``MASTER OF SCIENCE'' to ``DOCTOR OF
SCIENCE'' or ``DOCTOR OF PHILOSOPHY'', whichever applies.   \uline{IMPORTANT:
If you are a Ph.D.\ student,} you must also change the line above (near the
midsection of this page) to ``A dissertation presented to the Graduate School
of Arts and Sciences'' (but do NOT change the reference to the ``School of
Engineering'' which is at the very top of this page, as that must be left
exactly as shown.

{\small \textbf{Note for Ph.D.\ Students:}}
The formatting contained within this sample document can serve well in
emulating the basic formatting needed for the Ph.D.\ dissertation.   However,
please remember that all Ph.D.\ students are ultimately responsible for meeting
the Graduate School of Arts \& Sciences' formatting guidelines.  The GSAS
dissertation guidelines are published on the Graduate School web site located
with other documentation for GSAS policies and guides.  Be sure to read
``Important reminders'' in paragraph above.
\end{singlespace}
%\fi
\end{thesistitlepage}

\begin{thesiscopyrightpage}                 %% Generate the copyright page.
%\iffalse
\begin{singlespace}
\scriptsize
{\small \textbf{Important Notes Regarding Copyright Option:}} \\
Technically, a thesis or dissertation is protected to some degree by copyright
laws with or without a student having to register his or her claim to
copyright.  However, including a copyright page and applying for registration
of ones claim to copyright provide extra measures of legal protection from
potential copyright infringement.  There is a fee connected with explicitly
registering to copyright ones work; because of this, many students do not
choose to register to copyright their work.  Students should check with their
advisor(s) and/or seek legal advice to gather further information helpful to
making a decision with regards to registering their claim to copyright.  If you
are \uline{not} going to register to copyright your work, then you can choose
to remove this page from your document.  However, if you do choose to
explicitly copyright your work, then leave this page in, change the name to
your name, change the year to the appropriate year in which your degree will be
earned, and remove these notes of informational text.  If a student wishes to
officially ``register'' this claim to copyright, then Masters students will
need to pursue that effort on their own and can find appropriate options by
searching the web; Doctoral students can complete an authorization to apply for
registration (i.e., of their claim to copyright the dissertation) by indicating
this interest in the appropriate area of the UMI Dissertation Publishing
Agreement Form (i.e., on the form which they will submit along with their final
dissertation material) available from the Engineering Student Services web
site.

{\small \textbf{Important Notes Regarding Page Numbering and Margins:}} \\
If you decide to include this copyright page in your final document, do
\uline{not} count the page among your counted pages, and do \uline{not} display
any page number on the page.  \uline{Every sheet of paper in the manuscript
should be numbered except for two:  the title page and this optional copyright
page.}Specifically, the front textual information (which comes before your main
thesis/dissertation body of text) is numbered with Roman numerals, and your
main body of text begins with Arabic numbers.  Since the title page is counted
but \uline{not} numbered, roman numeral \uline{``ii'' is always the first
number used and appears on the page AFTER the title page (AND AFTER the
copyright page, IF included)} --- as shown in this sample template document.
Page numerals should always display centered, just above the 1 bottom margin.
The left margin should be 1.5 inches, with a 1 inch margin at top, bottom, and
right.  The left margin is extra-wide in order to accommodate the binding
process.  When typing the manuscript, stay well within these margin guides.
Lastly, remember to update your table of contents such that the page numerals
referenced there will match the page numbers on the bottom of the pages to
which they make reference in your document.  This is necessary to do manually
because, unfortunately, the page numbering within this templates table of
contents is \uline{not} automatically linked to the pages of the body of text.
This is further documented, along with some work arounds, in the appendix to
this guide called Special Notes for MS WORD Users.  \LaTeX{} users may have to
invent other solutions with regards to synchronizing table of contents page
references with actual document page numbers.  This guide merely provides a
helpful starting point.  \textbf{REMINDER:} When you remove these comments, be
sure to leave the copyright information centered both vertically and
horizontally on the page.
\end{singlespace}
%\fi
\end{thesiscopyrightpage}

\begin{thesisabstract}
\textbf{Reminders of what needs to be updated:}
After removing these comments, begin typing the body of your abstract here,
\uline{double-spaced}.  It is acceptable if the body of the abstract continues
onto the next page (as in this sample abstract), but \uline{the body of the
abstract is limited to a maximum of 350 words (excluding the heading
information listed above).  NOTE: This sample abstract is too long, as it
exceeds 350 words.}  The point-size of the body of the abstract can be set to
12 point (which is the text size of this sample comment-paragraph) or it can be
reduced to 10 point if you prefer.  Regardless of which specific point size you
select, the abstract must remain double-spaced and it should \uline{not} be
bolded.  \uline{If this is for your doctoral degree, be sure to change all
occurrences of the word ``thesis'' to display as ``dissertation'', and change
``Master of Science'' to ``Doctor of Science'' or ``Doctor of Philosophy'',
whichever applies.}  In the abstract heading above, make sure you \uline{use
the year your degree is officially to be earned}.  Be sure to use your full
name and your research advisor�s full name wherever appropriate, and be certain
to use the correct title of your degree whenever referencing it. The title of
your degree will not always be the same as the title of your department or
program, so please check with your departmental administrative assistant and
advisor(s) to be sure you are using the correct degree title.  Questions you
may have about preparing your theses or dissertations are always welcomed at
the Office of Engineering Student Services.

\textbf{Note for Ph.D.\ Students:}
The formatting contained within this sample document can serve well in
emulating the basic formatting needed for the Ph.D.\ dissertation.   However,
please remember that all Ph.D.\ students are ultimately responsible for meeting
the Graduate School of Arts \& Sciences' formatting guidelines.  The GSAS
thesis and dissertation guidelines are published on the Graduate School web
site located with other documentation for GSAS policies and guides.  Be sure to
read all of the above notes/reminders on what needs to be updated as shown in
this template document�s title, copyright, and abstract pages.  Ph.D.\ students
will submit final dissertations and all materials to the Office of Graduate
School of Arts and Sciences, and any questions about their dissertations should
also be directed to that office.
\end{thesisabstract}

%\iffalse
\renewcommand{\thesisacknowledgmentpostscript}{
\textbf{Reminders of what needs to be updated:}
After removing these comments, use the above format to help input your
acknowledgments page.   A special dedication can be placed as the final
paragraph, as shown above; alternatively, you may include a special dedication
on the page that follows, as also shown in this sample template.}
%\fi

\begin{thesisacknowledgments}
An acknowledgments page should be included in your final thesis or
dissertation.  In the final copy, it should be placed immediately before the
table of contents.  If you wish to include a special dedication, then you may
use the dedication to close the acknowledgments page or place it on the page
that immediately follows the acknowledgments page.  

It is appropriate to acknowledge sources of academic and financial support;
some fellowships and grants require acknowledgment.  Consequently, I would like
to thank the Dean for having the foresight and vision necessary to understand
the importance of funding the development of this sample thesis/dissertation
template.

A special thanks goes to the many graduate students and distinguished faculty
within my department who have reviewed this thesis and helped support the
related research.
\end{thesisacknowledgments}

\begin{thesisdedicationpage}                %% Generate the dedication page.
%\iffalse
\textbf{Note:} You may include a special dedication as shown here.  If you
include this page, be sure to keep it brief and center it on the page both
horizontally and vertically.  Alternatively, you may remove this page
altogether, and a special dedication can be placed as the final paragraph to
your acknowledgments page (as shown in this document on the preceeding page).
%\fi
\end{thesisdedicationpage}

\begin{singlespace}
\tableofcontents

%\iffalse
\renewcommand{\listoftablespostscript}{
\small
\textbf{Note:} Be consistent in aligning multi-lined table-names, figure-names,
and chapter/section-names throughout your document.  It is generally
recommended to make sure any additional lines (i.e., within a long title or a
long table name) wrap and align immediately under the 1st character of the
title or name with which they are associated in the line immediately above ---
as shown in the ``Table 2.1'' example above.   Whatever approach you take, be
consistent.}
%\fi

\listoftables

\listoffigures
\end{singlespace}

\chapter{Preface}

This guide contains the School of Engineering's rules for formatting theses and
dissertations.\footnote{Throughout this guide, the word thesis refers to both
theses and dissertations.}   Departments, advisors, and committees may impose
additional rules.  In the past, students were required to study a similar (but
much longer) set of rules and apply them to their theses.  The Association of
Graduate Engineering Students (i.e., AGES) has helped to prepare templates and
style files that simplify thesis preparation.  These files have been set up to
produce acceptably formatted theses and dissertations using several popular
word processing and text formatting programs.  There should be one available in
Microsoft WORD and another in \LaTeX{}.  Students can retrieve these files and
their accompanying instructions from the Engineering Student Services' main web
page.  Check with Engineering Student Servcies (Lopata Hall, Room 303) if you
have any questions.  Students who create their own templates or style files are
invited to submit these files for future use by others.  This guide you are
now reading can be downloaded (in either MS WORD formatted version or a \LaTeX{}
version) and can be utilized as a template for formatting your own theses.  In
short, the margin settings, pagination, table of contents logic, etc. are
already established in the downloadable versions.  You can simply replace the
text within the template with your own text, thereby saving you much setup
time.  \textbf{NOTE:} \uline{This preface page is optional.  A preface page
is usually used to explain further details surrounding the background and
motivation for the work.  You can remove it completely}, but then be sure the
reference to this page is also removed from the Table of Contents.  The
majority of students do not include a preface page.

%%
%% For List of Abbreviations, Glossary or Nomenclature also
%% use \chapter, but put some kind of list environment inside.



%%% Local Variables: 
%%% mode: latex
%%% TeX-master: "thesis-main"
%%% End: 
